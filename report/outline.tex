\documentclass[18pt, letter]{article}
\usepackage{graphicx} % Required for inserting images
\usepackage[bottom=1in, top=1in, left=1in, right=1in]{geometry}

\begin{document}
\thispagestyle{empty}
\begin{center}
    \Large{\textbf{SPACE PROPULSION PROJECT OUTLINE}}
\end{center}

\vspace{20pt}



\begin{center}
    \section*{Quad Chart Outline}
\end{center}

Project Title: Physics-Based Engine for Orbital Mechanics \\

Affiliation and Contacts: Concordia University \\


\begin{table}[h]
    \centering
    \begin{tabular}{|p{0.45\linewidth}|p{0.45\linewidth}|}
    \hline
    \textbf{Images of the Concept or something} & \textbf{Problem Description, Objectives, and Proposed New Idea or Technology}
 \newline 
    Objectives: Develop a physics-based engine to study and enhance spacecraft propulsion for orbital maneuvers.\\
    \hline

    
    \textbf{Technical Approach and Applications} 
    \newline Technical Approach: Utilizing C++ programming to simulate the physics-based engine's behavior in various orbital scenarios.
\newline Actions Done to Date: Developed initial algorithms and conducted preliminary simulations to validate the concept.
\newline Advantages: Enhanced efficiency and maneuverability, potentially enabling more precise orbital maneuvers.
 & \textbf{Challenges, Schedule, and Deliverables} \newline Overcoming computational complexity, optimizing algorithm efficiency, and ensuring accuracy of simulations. \newline Developed software package. \\
    \hline
    \end{tabular}
\end{table}


\clearpage
\begin{center}
    \section*{Technical Report Outline}
\end{center}

\begin{enumerate}
    \item INTRODUCTION 
    \begin{itemize}
        \item Research objectives
        \item Report Structure
    \end{itemize}

    \item Literature Review/State of the Arts % To be revised
    \begin{itemize}
        \item Examination of existing methods and models for orbital mechanics
        \item Review of how game engines use physics based models for simulations
        \item Implementation of orbital mechanics equations in computer language
    \end{itemize}

    \item Theory
    \begin{itemize}
        \item Equations governing the motion of objects in space and their interaction with other objects (based on two body universe theory)
        \item Briefly discuss fundamental principles of orbital mechanics
        \item Description of physics-based modeling
        \item Theoretical framework
    \end{itemize}

    \item Design and Development
    \begin{itemize}
        \item Overview of code structure
        \item Explanation of algorithms/equations used
        \item Challenges faced
    \end{itemize}

    \item Results
    \begin{itemize}
        \item Present results obtained from the code
        \item Brief explanation of the results and the objectives 
        \item  User instruction for the software
    \end{itemize}

    \item Discussion
    \begin{itemize}
        \item Assessment of the feasibility and practicality of this project
        \item Potential applications and implications for space exploration
        \item Limitations ad error analysis
    \end{itemize}

    \item Conslusion
    \begin{itemize}
        \item Summary and key findings
        \item Lessons learned
        \item Project potential
    \end{itemize}
    
\end{enumerate}
\clearpage







\end{document}
